\documentclass[a4paper, 10pt]{article}

\usepackage[T1]{fontenc}
\usepackage[utf8]{inputenc}
\usepackage[left=1.8cm, top=1.8cm, total={18cm, 25cm}]{geometry}
\usepackage[unicode]{hyperref}
\usepackage[slovak]{babel}

\title{Implementačná dokumentácia k 1. úlohe do IPP 2020/2021}
\author{Meno a priezvisko: Jakub Bartko \\ Login: xbartk07}
\date{}

\begin{document}
\maketitle

\section{Štruktúra kódu}
Činnosť skriptu \texttt{parser.php} je možné rozdeliť na spracovanie argumentov príkazového riadka, načítavanie zdrojového kódu, lexikálnu a syntaktickú analýzu inštrukcií, generovanie výstupu vo formáte XML a zbieranie štatistík o zdrojovom kóde. Hlavné súčasti týchto činností sú rozdelené na niekoľko hlavných funkcií, ktoré ďalej využívajú pomocné obslužné funkcie a štruktúry.

Medzi tieto štruktúry patria:
\begin{itemize}
\item trieda \texttt{State} --- jednoduchý zoznam stavov konečného automatu, ktorým je parser riadený
\item trieda \texttt{Counters} --- počítadlá riadkov kódu, komentárov, \dots a pomocné zoznamy návestí; slúžiace na zber štatistík pre rozšírenie \textsc{STATP}
\item asociatívne pole \texttt{opcodes} --- zoznam platných inštrukcií zdrojového kódu a typov ich argumentov (\texttt{variable}, \texttt{symbol} alebo \texttt{label})
\item asociatívne pole \texttt{opts} --- zoznam zadaných súborov štatistík a príslušných cieľových súborov
\end{itemize}
Okrem toho je výstup vo formáte XML zbieraný v premennej \texttt{[string] outputs} v rôznych bodoch parsovania.

\section{Činnosť parsera}
\textbf{Argumenty príkazového riadka} sú spracované iterovaním, kontrolou správnosti s využitím regulárnych výrazov a generovaním asociatívneho poľa \texttt{opts} pre príslušné súbory štatistík.

\textbf{Parsovanie zdrojového kódu} je v základe štruktúrované ako primitívny konečný automat. V každom svojom cykle načíta jeden riadok zo štandardného vstupu, odstráni z neho komentáre a riadok preskočí, ak obsahuje výhradne biele znaky. Ako prvý neprázdny riadok musí spracovať hlavičku zdrojového kódu; ďalej spracováva inštrukcie a ich argumenty.

\textbf{Spracovanie inštrukcie} pozostáva z kontroly jej operačného kódu a kontroly počtu a typu jej argumentov podľa referenčného asociatívneho poľa \texttt{opcodes}, a generovania XML elementu \texttt{instruction}. Obidve z uvedených kontrol sú vykonané s využitím regulárnych výrazov na základe definovaného typu: \texttt{variable}, \texttt{symbol} alebo \texttt{label}. Pri type \texttt{variable} sa skontroluje platnosť označenia rámca a formát identifikátora; pri type \texttt{symbol} sa skontroluje možnosť uvedenia premennej (kontrola identická s typom \texttt{variable}) a uvedenia konštanty, pri ktorej sa skontroluje jej špecifikovaný typ a platnosť príslušnej hodnoty. Okrem toho sa v reťazci, ktorý predstavuje argument inštrukcie, nahradia problémové znaky (pre formát XML) za zodpovedajúce XML entity.

\section{Rozšírenie \textsc{STATP}}
Štatistiky o zdrojovom kóde sú zbierané na príslušných miestach v skripte --- napr. počet riadkov kódu pri prečítaní nového neprázdneho riadku zdrojového kódu zo štandardného vstupu, alebo počet definovaných návestí pri parsovaní inštrukcie s operačným kódom \texttt{LABEL}. Dodatočné spracovanie si však vyžaduje vyhodnotenie počtu rôznych typov skokov (skokov dopredu, dozadu, alebo neplatných skokov). Toto vyhodnotenie je zabezpečené udržiavaním \textit{zoznamu definovaných návestí} a \textit{zoznamu destinácií skokov}. Pri inštrukcií s operačným kódom \texttt{LABEL} sa definované návestie uloží do prvého z týchto zoznamov a v prípade, že sa vyskytuje v zozname druhom, pripočíta sa ako \textit{skok dopredu}. Analogicky sa destinácia skoku pripočíta ako \textit{skok dozadu}, ak pri inštrukcií podmieneného alebo nepodmieného skoku špecifikované návestie už nachádza v zozname definovaných návestí. Nájdené cieľové návestia sa z príslušného zoznamu odstránia a tie, ktoré v ňom zostanú po prejdení celým zdrojovým programom sú považované za \texttt{bad jumps}. Na záver parsovania je zoznam \texttt{opts} iterovaný a špecifikované súbory štatistík sú zapísané do príslušných súborov.
\end{document}
